\documentclass[11pt]{article}

\usepackage{graphicx}
\usepackage{amsmath,amsfonts,amssymb}

\usepackage{hyperref}  % for urls and hyperlinks


\setlength{\textwidth}{6.2in}
\setlength{\oddsidemargin}{0.3in}
\setlength{\evensidemargin}{0in}
\setlength{\textheight}{8.9in}
\setlength{\voffset}{-1in}
\setlength{\headsep}{26pt}
\setlength{\parindent}{0pt}
\setlength{\parskip}{5pt}

% input some useful macros from RJLmacros.tex:
\input{../hw1/RJLmacros}

\begin{document}

\hfill\vbox{\hbox{AMath 574}\hbox{Homework 4}
\hbox{Due by 11:00pm on February 26, 2015}}

For submission instructions, see:

\url{http://faculty.washington.edu/rjl/classes/am574w2015/homework4.html}


%--------------------------------------------------------------------------
\vskip 1cm
\hrule
{\bf Problem 1.}

Consider the scalar conservation law $q_t + f(q)_x = 0$ with $f(q) =
\sqrt{q}$, which is convex as long as we only consider states $q > 0$.

{\bf 1(a).} Consider this problem with data

\begin{equation}\label{ic1}
q(x,0) = \bpwdef 4 \when 0 < x < 1,\\
                 1 \otherwise. \epwdef
\end{equation}

Determine the time $t_s$ when the shock and rarefaction wave first begin to
interact and the solution $q(x,t_s)$.

{\bf 1(b).} Use Clawpack to verify your solution.  Set up a problem similar
to what you did on the Programming problem of Homework \#3.  Choose the
domain, mesh size, and limiter to illustrate the solution in a convincing
manner.  Save the code required in a directory \verb+hw4/sqrtflux+ and add
a file \verb+README.txt+ with any instructions or comments on the solution.

{\bf 1(c).} Repeat parts (a) and (b) for the initial data

\begin{equation}\label{ic2}
q(x,0) = \bpwdef 4 \when 0 < x < 1,\\
                 0.01 \otherwise. \epwdef
\end{equation}

For the programming part, also do the following:

\begin{itemize}
\item Set \verb+clawdata.verbosity = 1+ in \verb+setrun.py+ so that it
prints out information every time step about the size of the time step.
Comment on what you observe relative to a similar experiment with the
initial data \eqn{ic1}.

\item Try using the Lax-Wendroff method (no limiter) for this problem and
comment on what happens, relative to a similar experiment with the
initial data \eqn{ic1}.
\end{itemize}

If the results to the second part are mysterious, here are some hints:

\begin{itemize}
\item You might check the file \verb+_output/fort.q0001+ to see 
what the values the solution takes.
\item In \verb+setrun.py+ if you set \verb+clawdata.output_style = 1+, you
can have it output results every time step for a certain number of steps.
\item Fortran does not always trap arithmetic exceptions unless you tell it
to, for example by setting \\
\verb+  FFLAGS =  -ffpe-trap=invalid,overflow,zero+\\
in the Makefile and then doing \verb+make new+ to recompile everything.
(These flags are for \verb+gfortran+.) 

See \url{http://www.clawpack.org/fortran_compilers.html} for some other
useful debugging flags.
\end{itemize} 

You do not need to check in code for Problem 1(c), but include some
discussion in your written solutions about what you observe, including
figures if you'd like.


Make sure you have committed any files to your repository that are
needed to run your code and produce the plots.
You do {\bf not} need to commit the output files or plots, or the files
produced by compiling the fortran code (\verb+*.o, xclaw+).


% uncomment the next two lines if you want to insert solution...
%\vskip 1cm
%{\bf Solution:}

% insert your solution here!

%--------------------------------------------------------------------------
\vskip 1cm
\hrule
{\bf Problem 2.}

Consider the linear system $q_t + Aq_x = 0$ with 
\[
A = \brm -1 & 1\\ 0 & 2\erm.
\]

{\bf 2(a).} Determine the eigenvalues and eigenvectors of $A$ and sketch the
integral curves of the eigenvectors in the phase plane.
(And recall that for a linear system these are also the Hugoniot loci.)

{\bf 2(b).} Consider the Cauchy problem for this system (no boundaries) with
initial data
\[
q(x,0) = \bpwdef ~[4,4]^T &\text{if}~~ x < 1,\\
                 ~[1,1]^T &\text{if}~~ 1 \leq x \leq 3,\\
                 ~[2,1]^T &\text{if}~~ x > 3.\epwdef
\]

{\bf 2(c).} Set up a Clawpack Riemann solver for this problem and use the
initial condition above as a test of your code.  
(You might want to use the code in
\verb+$CLAW/classic/examples/acoustics_1d_example1+ as a model for a linear 
system of two equations.)
Solve it over a large
enough domain to see all the states you expect to see in the exact solution,
and use extrapolation boundary conditions.   

Include some plots from your solution in your writeup.

Commit the files needed to produce these plots, in a directory
\verb+hw4/linsys+.

Please make sure comments in the code (and labels on plots) 
are relevant to the problem being solved and clean up things not needed for
this code (e.g. parts of the acoustics code dealing with the physical
parameters that are not relevant to this problem).

% uncomment the next two lines if you want to insert solution...
%\vskip 1cm
%{\bf Solution:}

% insert your solution here!


%--------------------------------------------------------------------------
\vskip 1cm
\hrule
{\bf Problem 13.11 in the book, parts (a)-(c).}

Also think about how the Riemann solution looks in the phase plane ($q$--$u$
plane in this case) and think
about part (d) in this context.  You don't need to turn in anything, but try
to understand the phase plane for this problem.


% uncomment the next two lines if you want to insert solution...
%\vskip 1cm
%{\bf Solution:}

% insert your solution here!


%--------------------------------------------------------------------------
\vskip 1cm
\hrule
{\bf Problem 15.2 in the book.}


% uncomment the next two lines if you want to insert solution...
%\vskip 1cm
%{\bf Solution:}

% insert your solution here!


\end{document}


