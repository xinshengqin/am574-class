\documentclass[11pt]{article}

\usepackage{graphicx}
\usepackage{amsmath,amsfonts,amssymb}

\usepackage{hyperref}  % for urls and hyperlinks


\setlength{\textwidth}{6.2in}
\setlength{\oddsidemargin}{0.3in}
\setlength{\evensidemargin}{0in}
\setlength{\textheight}{8.9in}
\setlength{\voffset}{-1in}
\setlength{\headsep}{26pt}
\setlength{\parindent}{0pt}
\setlength{\parskip}{5pt}

% input some useful macros from RJLmacros.tex:
\input{../hw1/RJLmacros}

\begin{document}

\hfill\vbox{\hbox{AMath 574}\hbox{Homework 2}
\hbox{Due by 11:00pm on January 29, 2015}}

For submission instructions, see:

\url{http://faculty.washington.edu/rjl/classes/am574w2015/homework2.html}

%--------------------------------------------------------------------------
\vskip 1cm
\hrule
{\bf Problem \#4.1 in the book}


% uncomment the next two lines if you want to insert solution...
%\vskip 1cm
%{\bf Solution:}

% insert your solution here!

%--------------------------------------------------------------------------

%--------------------------------------------------------------------------
\vskip 1cm
\hrule
{\bf Problem \#4.2 in the book}


% uncomment the next two lines if you want to insert solution...
%\vskip 1cm
%{\bf Solution:}

% insert your solution here!

%--------------------------------------------------------------------------
\vskip 1cm
\hrule
{\bf Problem \#6.1 in the book}


% uncomment the next two lines if you want to insert solution...
%\vskip 1cm
%{\bf Solution:}

% insert your solution here!

%--------------------------------------------------------------------------
\vskip 1cm
\hrule
{\bf Problem \#6.2 in the book}


% uncomment the next two lines if you want to insert solution...
%\vskip 1cm
%{\bf Solution:}

% insert your solution here!

%--------------------------------------------------------------------------
\vskip 1cm
\hrule
{\bf Problem \#6.3 in the book}


% uncomment the next two lines if you want to insert solution...
%\vskip 1cm
%{\bf Solution:}

% insert your solution here!

%--------------------------------------------------------------------------

\vskip 1cm
\hrule
{\bf Problem \#6.4 in the book}


% uncomment the next two lines if you want to insert solution...
%\vskip 1cm
%{\bf Solution:}

% insert your solution here!


%--------------------------------------------------------------------------

\vskip 1cm
\hrule
{\bf Problem \#6.5 in the book}


% uncomment the next two lines if you want to insert solution...
%\vskip 1cm
%{\bf Solution:}

% insert your solution here!


%--------------------------------------------------------------------------
\vskip 1cm
\hrule
{\bf Problem \#6.6 in the book}

% uncomment the next two lines if you want to insert solution...
%\vskip 1cm
%{\bf Solution:}

% insert your solution here!

%--------------------------------------------------------------------------

\vskip 1cm
\hrule
{\bf Additional Problem:}

Show that the flux-limiter method (6.40),(6.41) can be written as a {\em
wave limiter} method as:

\[
Q_i^{n+1} = Q_i^n - \dtdx (\bar u^+ \W_{i-1/2}  + \bar u^- \W_{i+1/2})
-\dtdx (\widetilde F_{i+1/2} - \widetilde F_{i-1/2}),
\]
where $\W_{i-1/2} = Q_i^n - Q_{i-1}^n$ and the ``correction flux'' is
\[
\widetilde F_{i-1/2} = \half |\bar u| 
\left(1 - \dtdx |\bar u|\right) \widetilde \W_{i-1/2},
\]
with the limited waves $\widetilde \W$ defined by
\[
\widetilde \W_{i-1/2} = \phi(\theta_{i-1/2}) \W_{i-1/2}.
\]
The ratio $\theta_{i-1/2}$ is defined in (6.35) and the
function $\phi$ might be one of limiters from (6.39).

% uncomment the next two lines if you want to insert solution...
%\vskip 1cm
%{\bf Solution:}

% insert your solution here!

%--------------------------------------------------------------------------
\newpage

\vskip 1cm
\hrule
{\bf Programming problem}

Modify the IPython notebook \verb+$AM574/labs/lab2/AdvectionTests.ipynb+
to create a new notebook \verb+AdvectionLimiters.ipynb+ that illustrates
your solutions to the following:

\renewcommand{\theenumi}{\alph{enumi}}  % use letters for enumeration
\begin{enumerate}
\item Modify the upwind code to use two ghost cells on each side rather than
one.  Check that this gives identical results to the original code for cases
when the time {\tt tfinal} is large enough that the periodic boundary
conditions play a role.  Include at least one of these tests in the notebook.
Note that the method implemented in the next part will require two ghost cells.

\item Implement the wave limiter methods for advection, as described in 
the previous problem.   Note that it's impossible to use half-integer
indices, so you might want to declare arrays such as \verb+Ftilde+ in which 
\verb+Ftilde[i]+ holds the correction flux $\widetilde F_{i-1/2}$.  (This is
the convention used in Clawpack --- the index $i$ often refers to
information at the left edge of the cell $x_{i-1/2}$.)

Copy the \verb+upwind+ function definition to a new cell in the notebook and
modify it to create a new function \verb+wave_limiter+ that has one
additional argument \verb+phi+ in the calling sequence, so that a limiter
function $\phi(\theta)$ can be passed in.  The function $\phi$ might be one
of those listed in (6.39a,b) in the book.  

For example:
\begin{verbatim}
    def phi_minmod(theta):
        return(max(0., min(theta,1)))
\end{verbatim}
would define the minmax limiter, and then
\begin{verbatim}
    q = wave_limiter(ubar,q0,x,tfinal,nsteps,phi_minmod)
\end{verbatim}
should solve the problem using the minmod wave-limiter method.

\item Test your function by reproducing
Figures 6.2 and 6.3 from the book in your notebook.

\end{enumerate} 

Submit your notebook \verb+AdvectionLimiters.ipynb+.

\end{document}

